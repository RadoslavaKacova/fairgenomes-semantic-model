\documentclass{article}
\usepackage[utf8]{inputenc}
\begin{document}
\setlength\parindent{0pt}

\textbf{FAIR Genomes semantic metadata schema}
\newline

The FAIR Genomes semantic metadata schema to power reuse of NGS data in research and healthcare. Version 1.1-Minor, 2021-07-20. This model consists of 9 modules that contain 110 metadata elements and 85307 lookups in total (excluding null flavors).

\begin{table}[htb]
\begin{tabular}{lll}
Name & Ontology & Nr. of elements \\
\hline
Study & NCIT:C63536 & 9 \\
Personal & NCIT:C90492 & 12 \\
Leaflet and consent form & NCIT:C16468 & 9 \\
Individual consent & NCIT:C16735 & 12 \\
Clinical & NCIT:C25398 & 20 \\
Material & NCIT:C43376 & 16 \\
Sample preparation & OBI:0001902 & 9 \\
Sequencing & EDAM:topic\_3168 & 12 \\
Analysis & EDAM:operation\_2945 & 11 \\
\hline
\end{tabular}
\caption[Module overview]{\label{table:table1} FAIR Genomes v1.1-Minor overview of all modules.}
\end{table}

\begin{table}[htb]
\begin{tabular}{lll}
Name & Ontology & Values \\
\hline
Identifier & OMIABIS:0000006 & UniqueID \\
Name & OMIABIS:0000037 & String \\
Description & OMIABIS:0000036 & Text \\
Inclusion criteria & OBI:0500027 & InclusionCriteria lookup (14 choices) \\
Principal investigator & OMIABIS:0000100 & String \\
Contact information & OMIABIS:0000035 & String \\
Study design & OBI:0500000 & Text \\
Start date & NCIT:C69208 & Date \\
Completion date & NCIT:C142702 & Date \\
\hline
\end{tabular}
\caption[Module: Study]{\label{table:table2} Module: Study. A detailed examination, analysis, or critical inspection of one or multiple subjects designed to discover facts. Ontology: NCIT:C63536. }
\end{table}

\begin{table}[htb]
\begin{tabular}{lll}
Name & Ontology & Values \\
\hline
Personal identifier & NCIT:C164337 & UniqueID \\
Phenotypic sex & PATO:0001894 & PhenotypicSex lookup (4 choices) \\
Genotypic sex & PATO:0020000 & GenotypicSex lookup (12 choices) \\
Country of residence & NCIT:C171105 & Countries lookup (249 choices) \\
Ancestry & NCIT:C176763 & Ancestry lookup (305 choices) \\
Country of birth & GENEPIO:0001094 & Countries lookup (249 choices) \\
Year of birth & NCIT:C83164 & Integer \\
Inclusion status & NCIT:C166244 & InclusionStatus lookup (4 choices) \\
Age at death & NCIT:C135383 & Integer \\
Primary affiliated institute & NCIT:C25412 & Institutes lookup (218 choices) \\
Resources in other institutes & NCIT:C19012 & Institutes lookup (218 choices) \\
Participates in study & RO:0000056 & Reference to Study \\
\hline
\end{tabular}
\caption[Module: Personal]{\label{table:table3} Module: Personal. Data, facts or figures about an individual; the set of relevant items would depend on the use case. Ontology: NCIT:C90492. }
\end{table}

\begin{table}[htb]
\begin{tabular}{lll}
Name & Ontology & Values \\
\hline
Leaflet title & DC:title & String \\
Leaflet date & DC:date & Date \\
Leaflet version & DC:hasVersion & String \\
Consent form identifier & DC:identifier & UniqueID \\
Consent form title & DC:title & String \\
Consent form accepted date & DC:dateAccepted & Date \\
Consent form valid until & DC:valid & Date \\
Consent form creator & NCIT:C42628 & Institutes lookup (218 choices) \\
Consent form version & DC:hasVersion & String \\
\hline
\end{tabular}
\caption[Module: Leaflet and consent form]{\label{table:table4} Module: Leaflet and consent form. A document explaining all the relevant information to assist an individual in understanding the expectations and risks in making a decision about a procedure. This document is presented to and signed by the individual or guardian. Ontology: NCIT:C16468. }
\end{table}

\begin{table}[htb]
\begin{tabular}{lll}
Name & Ontology & Values \\
\hline
Individual consent identifier & ICO:0000044 & UniqueID \\
Person consenting & IAO:0000136 & Reference to Personal \\
Consent form used & IAO:0000136 & Reference to Leaflet and consent form \\
Collected by & NCIT:C45262 & Institutes lookup (218 choices) \\
Signing date & ICO:0000036 & Date \\
Valid from & DC:valid & Date \\
Valid until & DC:valid & Date \\
Represented by & NCIT:C142600 & RepresentedBy lookup (3 choices) \\
Data use permissions & DUO:0000001 & DataUsePermissions lookup (5 choices) \\
Data use modifiers & DUO:0000017 & DataUseModifiers lookup (23 choices) \\
Data use specification & SIO:000090 & Text \\
Allow recontacting & NCIT:C25737 & Recontacting lookup (3 choices) \\
\hline
\end{tabular}
\caption[Module: Individual consent]{\label{table:table5} Module: Individual consent. Consent given by a patient to a surgical or medical procedure or participation in a study, examination or analysis after achieving an understanding of the relevant medical facts and the risks involved. Ontology: NCIT:C16735. }
\end{table}

\begin{table}[htb]
\begin{tabular}{lll}
Name & Ontology & Values \\
\hline
Clinical identifier & NCIT:C87853 & UniqueID \\
Belongs to person & IAO:0000136 & Reference to Personal \\
Phenotype & NCIT:C16977 & Phenotypes lookup (15802 choices) \\
Unobserved phenotype & HL7:C0442737 & Phenotypes lookup (15802 choices) \\
Phenotypic data available & NCIT:C15783 & DCMITypes lookup (6 choices) \\
Clinical diagnosis & NCIT:C15607 & Diseases lookup (9700 choices) \\
Molecular diagnosis gene & NCIT:C20826 & Genes lookup (19202 choices) \\
Molecular diagnosis other & NCIT:C20826 & Text \\
Age at diagnosis & SNOMEDCT:423493009 & Integer \\
Age at last screening & NCIT:C81258 & Integer \\
Medication & NCIT:C459 & Drugs lookup (5632 choices) \\
Drug regimen & NCIT:C142516 & Text \\
Family members affected & HP:0032320 & FamilyMembers lookup (41 choices) \\
Family members sequenced & NCIT:C79916 & FamilyMembers lookup (41 choices) \\
Consanguinity & GSSO:007578 & String \\
Medical history & NCIT:C18772 & MedicalHistory lookup (1154 choices) \\
Age of onset & Orphanet:C023 & Integer \\
First contact & LOINC:MTHU048806 & Date \\
Functioning & NCIT:C21007 & Text \\
Material used in diagnosis & SIO:000641 & String \\
\hline
\end{tabular}
\caption[Module: Clinical]{\label{table:table6} Module: Clinical. Findings and circumstances relating to the examination and treatment of a patient. Ontology: NCIT:C25398. }
\end{table}

\begin{table}[htb]
\begin{tabular}{lll}
Name & Ontology & Values \\
\hline
Material identifier & NCIT:C93400 & UniqueID \\
Collected from person & SIO:000244 & Reference to Personal \\
Belongs to diagnosis & SIO:000068 & Reference to Clinical \\
Sampling timestamp & EFO:0000689 & DateTime \\
Registration timestamp & NCIT:C25646 & DateTime \\
Sampling protocol & EFO:0005518 & Text \\
Sampling protocol deviation & NCIT:C50996 & String \\
Reason for sampling protocol deviation & NCIT:C93529 & String \\
Biospecimen type & NCIT:C70713 & BiospecimenTypes lookup (403 choices) \\
Anatomical source & NCIT:C103264 & AnatomicalSources lookup (13827 choices) \\
Pathological state & GO:0001894 & PathologicalState lookup (4 choices) \\
Storage conditions & NCIT:C96145 & StorageConditions lookup (26 choices) \\
Expiration date & NCIT:C164516 & Date \\
Percentage tumor cells & NCIT:C127771 & Decimal \\
Physical location & GAZ:00000448 & String \\
Derived from & NCIT:C28355 & String \\
\hline
\end{tabular}
\caption[Module: Material]{\label{table:table7} Module: Material. A natural substance derived from living organisms such as cells, tissues, proteins, and DNA. Ontology: NCIT:C43376. }
\end{table}

\begin{table}[htb]
\begin{tabular}{lll}
Name & Ontology & Values \\
\hline
Sampleprep identifier & NCIT:C132299 & UniqueID \\
Belongs to material & NCIT:C25683 & Reference to Material \\
Input amount & AFRL:0000010 & Integer \\
Library preparation kit & GENEPIO:0000085 & NGSKits lookup (616 choices) \\
PCR free & NCIT:C17003 & Boolean \\
Target enrichment kit & NCIT:C154307 & NGSKits lookup (616 choices) \\
UMIs present & EFO:0010199 & Boolean \\
Intended insert size & FG:0000001 & Integer \\
Intended read length & NCIT:C153362 & Integer \\
\hline
\end{tabular}
\caption[Module: Sample preparation]{\label{table:table8} Module: Sample preparation. A sample preparation for a nucleic acids sequencing assay. Ontology: OBI:0001902. }
\end{table}

\begin{table}[htb]
\begin{tabular}{lll}
Name & Ontology & Values \\
\hline
Sequencing identifier & NCIT:C171337 & UniqueID \\
Belongs to sample & NCIT:C25683 & Reference to Sample preparation \\
Sequencing date & GENEPIO:0000069 & Date \\
Sequencing platform & GENEPIO:0000071 & SequencingPlatform lookup (7 choices) \\
Sequencing instrument model & GENEPIO:0001921 & SequencingInstrumentModels lookup (39 choices) \\
Sequencing method & FIX:0000704 & SequencingMethods lookup (35 choices) \\
Average read depth & NCIT:C155320 & Integer \\
Observed read length & NCIT:C153362 & Integer \\
Observed insert size & FG:0000002 & Integer \\
Percentage Q30 & GENEPIO:0000089 & Decimal \\
Percentage TR20 & FG:0000003 & Decimal \\
Other quality metrics & EDAM:data\_3914 & Text \\
\hline
\end{tabular}
\caption[Module: Sequencing]{\label{table:table9} Module: Sequencing. The determination of complete (typically nucleotide) sequences, including those of genomes (full genome sequencing, de novo sequencing and resequencing), amplicons and transcriptomes. Ontology: EDAM:topic\_3168. }
\end{table}

\begin{table}[htb]
\begin{tabular}{lll}
Name & Ontology & Values \\
\hline
Analysis identifier & AFR:0001979 & UniqueID \\
Belongs to sequencing & NCIT:C25683 & Reference to Sequencing \\
Physical data location & GAZ:00000448 & String \\
Abstract data location & NCIT:C142494 & String \\
Data formats stored & NCIT:C142494 & DataFormats lookup (582 choices) \\
Algorithms used & NCIT:C16275 & Text \\
Reference genome used & EDAM:data\_2340 & GenomeAccessions lookup (29 choices) \\
Bioinformatic protocol used & EDAM:data\_2531 & Text \\
Bioinformatic protocol deviation & NCIT:C50996 & String \\
Reason for bioinformatic protocol deviation & NCIT:C93529 & String \\
WGS guideline followed & NCIT:C17564 & String \\
\hline
\end{tabular}
\caption[Module: Analysis]{\label{table:table10} Module: Analysis. An analysis applies analytical (often computational) methods to existing data of a specific type to produce some desired output. Ontology: EDAM:operation\_2945. }
\end{table}

\begin{table}[htb]
\begin{tabular}{ll}
Value & Ontology \\
\hline
NoInformation & HL7:NI \\
Invalid & HL7:INV \\
Derived & HL7:DER \\
Other & HL7:OTH \\
Negative infinity & HL7:NINF \\
Positive infinity & HL7:PINF \\
Un-encoded & HL7:UNC \\
Masked & HL7:MSK \\
Not applicable & HL7:NA \\
Unknown & HL7:UNK \\
Asked but unknown & HL7:ASKU \\
Temporarily unavailable & HL7:NAV \\
Not asked & HL7:NASK \\
Not available & HL7:NAVU \\
Sufficient quantity & HL7:QS \\
Trace & HL7:TRC \\
\hline
\end{tabular}
\caption[NullFlavors]{\label{table:table11} Overview of null flavors. Each lookup in FAIR Genomes is supplemented with so-called 'null flavors' from HL7. These can be used to indicate precisely why a particular value could not be entered into the system, providing substantially more insight than simply leaving a field empty. }
\end{table}

\end{document}
